\documentclass[10pt, twocolumn]{article}

\usepackage[a4paper, margin=0.5in]{geometry}
\usepackage{amsfonts}
\usepackage{amsthm}
\usepackage{amsmath}
\usepackage{parskip}
\usepackage{mathtools}
\usepackage{mathrsfs}
\usepackage{graphicx}
\usepackage{amssymb}
\usepackage{xtab}
\usepackage{physics}
\usepackage{algorithm}
\usepackage{algorithmic}
\usepackage{seqsplit}
\usepackage{enumitem}
\usepackage{tikz}
\usepackage{braket}
\usepackage{dirtytalk}
\usepackage{hyperref}
\usetikzlibrary{quantikz}

\DeclareMathOperator{\lcm}{lcm}
\DeclarePairedDelimiter{\floor}{\lfloor}{\rfloor}
\DeclarePairedDelimiter{\ceil}{\lceil}{\rceil}
\DeclareMathOperator*{\argmax}{arg\,max}
\DeclareMathOperator*{\argmin}{arg\,min}

\newcommand*\diff{\mathop{}\!\mathrm{d}}

\newtheorem{theorem}{Theorem}[section]
\newtheorem{definition}[theorem]{Definition}
\newtheorem{proposition}[theorem]{Proposition}
\newtheorem{corollary}[theorem]{Corollary}
\newtheorem{lemma}[theorem]{Lemma}
\newtheorem{postulate}[theorem]{Postulate}
\newtheorem{problem}[theorem]{Problem}
\newtheorem*{claim}{Claim}


\usepackage[backend=biber, sorting=none]{biblatex}
\bibliography{bibliography}

\begin{document}

\title{QCE'24 Quantum Resource Estimation Educational Challenge - Matrix Inversion by QSVT}
\author{Walden Killick}

\maketitle

\begin{abstract}
	We produce physical resource estimates for solving systems of linear equations involving banded circulant matrices using the quantum singular value transformation algorithm.
\end{abstract}

\section{Introduction}

Since the breakthrough 2009 algorithm of Harrow, Hassidim, and Lloyd (HHL algorithm), it has been well-known that for sparse, well-conditioned matrices, quantum computers are capable of solving systems of linear equations (SLEs) exponentially faster than the best classical algorithms \cite{harrow2009quantum}. To date, the HHL algorithm is still the most commonly cited quantum algorithm for solving SLEs, both in academic research and for industry applications. On the other hand, subsequent improvements to the HHL algorithm have not received the same attention despite achieving super complexity and being arguably conceptually simpler. In this project, we consider the use of the quantum singular value transformation (QSVT) \cite{gilyen2019quantum, martyn2021grand} in solving systems of linear equations and investigate the physical resource requirements this entails.

All code used to generate the results in this report can  be found at \href{https://github.com/Walden-Killick/QCE24-QRE-Challenge}{https://github.com/Walden-Killick/QCE24-QRE-Challenge}.

\section{Preliminary}

\subsection{Problem definition}

Given a quantum state $\ket{b}$ and an efficient classical description of a sparse matrix $A$, the quantum linear system of equations (QSLE) problem is the problem of preparing the state $\ket{x} = A^{-1} \ket{b}$. Broadly speaking, the primary challenges in any QSLE algorithm are (1) accessing $A$ in a quantum circuit, and (2) performing (approximately) the transformation $x \mapsto 1/x$ on $A$'s eigenvalues. The HHL algorithm \cite{harrow2009quantum} achieves these goals using a sparse Hamiltonian simulation subroutine \cite{berry2007efficient} and quantum phase estimation \cite{kitaev1995quantum}, respectively. By contrast, QSVT utilises the fundamentally different techniques of block encoding \cite{gilyen2019quantum} and quantum signal processing \cite{low2017optimal} and achieves an exponentially improved dependence on the target precision as well as a lesser improvement with respect to the condtion number. We briefly review these techniques in the following subsection.

\subsection{Matrix inversion by QSVT}

To access $A$ in a quantum circuit, QSVT relies on quantum oracles which efficiently encode both the non-zero structure and non-zero elements of $A$. These are defined rigorously as follows.

\begin{definition}[Sparse access oracles \cite{camps2203explicit}]
	\label{def::sparse_access_oracles}
	Let $A$ be a $2^n \times 2^n$ $s$-sparse matrix. Let $c(j,l)$ be a function which returns the row index of the $l$-th non-zero matrix element in the $j$-th column of $A$. The sparse access oracles $O_c$ and $O_A$ for $A$ are the unitary operators defined as
	\[
		O_c \ket{l}\ket{j} = \ket{l}\ket{c(j, l)}
	\]
	and
	\[
		O_A \ket{0}\ket{l}\ket{j} = \left( A_{c(j, l), j} \ket{0} + \sqrt{1 - \abs{A_{c(j, l), j}}^2 } \ket{1} \right) \ket{l} \ket{j}.
	\]
\end{definition}

We should note that in this formalism, $O_A$ also implicitly encodes the non-zero structure of $A$ by calling $c(j, l)$; however, for certain sufficiently well-structured matrices such as those considered in this project, this is not prohibitive.

The following informal theorem states that one can construct a unitary matrix which directly contains (a scaled version of) $A$ as a submatrix (block).

\begin{theorem}[Block-encoding sparse matrices \cite{gilyen2019quantum, camps2203explicit}]
	\label{thm:block_encoding_sparse_matrices}
	Let $A, O_c, O_A$ be as in Definition \ref{def::sparse_access_oracles}. Using one call to each of $O_c$ and $O_A$, we can construct a larger unitary matrix $U_A$ for which the upper left $2^n \times 2^n$ block is equal to $A/s$.
\end{theorem}

The final ingredient in matrix inversion by QSVT is then to transform the singular values of $A/s$ as $x \mapsto 1/x$. QSVT cannot perform this transformation exactly, but can perform almost-arbitrary polynomial transformations \cite{sunderhauf2023generalized} and thus approximate $1/x$ to arbitrary precision.

\begin{theorem}[Matrix inversion by QSVT \cite{gilyen2019quantum}]
	Let $A$ be a sparse matrix with condition number $\kappa$ and let $U_A$ block-encode $A$ as in Theorem \ref{thm:block_encoding_sparse_matrices}. Using $O(\kappa \log(\kappa / \varepsilon))$ calls to $U_A$ and $U_A^\dag$, we can construct a unitary which block-encodes $A^{-1}$ to within accuracy $\varepsilon$.
\end{theorem}

We refer the reader to \cite{martyn2021grand} for a pedagogical introduction to block encodings and QSVT.

\subsection{Banded circulant matrices}

To construct polynomial-size oracles $O_c$ and $O_A$, $A$ must be well-structured in some way. For this project, we consider banded circulant matrices due to the known explicit and simple construction of their sparse access oracles \cite{camps2203explicit}.

\begin{definition}[Circulant matrix]
	A circulant matrix is a matrix in which each row is equal to the previous row but with each element shifted one place to the right.
\end{definition}

\begin{definition}[Banded circulant matrix]
	A circulant matrix is banded if only the diagonal, subdiagonal, and superdiagonal elements are non-zero.
	\[
		\begin{bmatrix}
			\alpha & \gamma & 0 & \cdots & \beta \\
			\beta & \alpha & \gamma & \cdots & 0 \\
			0 & \beta & \alpha & \cdots & 0 \\
			\vdots & \vdots & \vdots & \ddots & \vdots \\
			\gamma & 0 & 0 & \cdots & \alpha
		\end{bmatrix}
	\]
\end{definition}

\newpage

\printbibliography

\end{document}