\documentclass[11pt, twocolumn]{article}

\usepackage[a4paper, margin=1in]{geometry}
\usepackage{amsfonts}
\usepackage{amsthm}
\usepackage{amsmath}
\usepackage{parskip}
\usepackage{mathtools}
\usepackage{mathrsfs}
\usepackage{graphicx}
\usepackage{amssymb}
\usepackage{xtab}
\usepackage{physics}
\usepackage{algorithm}
\usepackage{algorithmic}
\usepackage{seqsplit}
\usepackage{enumitem}
\usepackage{tikz}
\usepackage{braket}
\usepackage{dirtytalk}
\usepackage{hyperref}
\usetikzlibrary{quantikz}

\usepackage[backend=biber, sorting=none]{biblatex}
\bibliography{bibliography}

\begin{document}

\title{QCE'24 Quantum Resource Estimation Educational Challenge - Matrix Inversion by QSVT}
\author{Walden Killick}

\maketitle

\begin{abstract}
	We perform resource estimation for solving systems of linear equations involving banded circulant matrices using the quantum singular value transformation algorithm.
\end{abstract}

\section{Introduction}

Since the breakthrough 2009 algorithm of Harrow, Hassidim, and Lloyd (HHL algorithm), it has been well-known that for sparse, well-conditioned matrices, quantum computers are capable of solving systems of linear equations (SLEs) exponentially faster than the best classical algorithms \cite{harrow2009quantum}. To date, the HHL algorithm is still the most commonly cited quantum algorithm for solving SLEs, both in academic research and for potential industry applications. On the other hand, subsequent improvements to the HHL algorithm have not received the same attention despite achieving super complexity and being arguably conceptually simpler \cite{martyn2021grand}. In this project, we consider the use of the quantum singular value transformation \cite{gilyen2019quantum} in solving systems of linear equations and investigate the physical resource requirements this entails.

All code used to generate the results in this report can  be found at \href{https://github.com/Walden-Killick/QCE24-QRE-Challenge}{https://github.com/Walden-Killick/QCE24-QRE-Challenge}.

\newpage

\printbibliography

\end{document}